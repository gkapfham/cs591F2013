\input{hwpre.tex}

\usepackage[compact]{titlesec}

\begin{document}
\MYTITLE{Homework Assignment One: Understanding and Customizing Android}
\MYHEADERS{Homework Assignment One}{}

\section*{Introduction}

Throughout this semester, we will use the Nexus 7 tablet and the Android operating system to learn more about mobile
applications and mobile computing.  In this homework assignment, we will learn some of the basics of the Android
operating system and explore several applications that can customize and extend our interactions with the Nexus 7 tablets.

\section*{Understanding Android}

In this class, we will use Android ``Jelly Bean'' 4.3.  To learn more about Android and the 4.3 version of this
operating system, you should visit and read the following Web sites: \url{http://www.android.com/} and
\url{http://www.android.com/whatsnew/} (students are encouraged to find and study other Android Web sites as well). Try
to find the user interface components and operating system features that these articles mention.  For instance, Android
has a notification area --- where is it?  How do you use it? Android also has a lock screen that supports the use of
widgets --- how do you add a widget to the lock screen?  Of course, there are many other important aspects of Android.
Make a list that gives a name and screenshot demonstrating each key aspect of Android.  What features of Android do you
find to be the most useful?  What features do you dislike?

Before we further explore and customize the Android environment, it is a good idea to test the tablets to ensure that
they are functioning correctly.  Using the Google Play store as the download source for these apps, please install the
``AndroSensor'', ``GPS Test'', ``PixelTest'', and ``YAMTT''.  Using at least one screenshot of each app, please explain
their inputs, outputs, and behavior.

\section*{Customizing Android}




\section*{Summary of the Required Deliverables}

This assignment invites your team to submit one printed version of a tutorial that contains:

\begin{enumerate}
	
	\item A description of the steps that a user must take to configure Git and Bitbucket

	\item A description of the inputs, outputs, and behavior of the aforementioned Git commands

\end{enumerate}

You must also ensure that the instructor has read access to your Bitbucket repository that is named according to the
convention {\tt cs290F2013-lab1-team{\em k}}, with {\tt {\em k}} representing the number of your assigned team. Please
see the instructor if you would like to print your tutorial slides in color.

\end{document}
