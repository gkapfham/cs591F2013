	% CS 580 style
% Typical usage (all UPPERCASE items are optional):
%       \input 580pre
%       \begin{document}
%       \MYTITLE{Title of document, e.g., Lab 1\\Due ...}
%       \MYHEADERS{short title}{other running head, e.g., due date}
%       \PURPOSE{Description of purpose}
%       \SUMMARY{Very short overview of assignment}
%       \DETAILS{Detailed description}
%         \SUBHEAD{if needed} ...
%         \SUBHEAD{if needed} ...
%          ...
%       \HANDIN{What to hand in and how}
%       \begin{checklist}
%       \item ...
%       \end{checklist}
% There is no need to include a "\documentstyle."
% However, there should be an "\end{document}."
%
%===========================================================
\documentclass[11pt,twoside,titlepage]{article}
%%NEED TO ADD epsf!!
\usepackage{threeparttop}
\usepackage{graphicx}
\usepackage{latexsym}
\usepackage{color}
\usepackage{listings}
\usepackage{fancyvrb}
%\usepackage{pgf,pgfarrows,pgfnodes,pgfautomata,pgfheaps,pgfshade}
\usepackage{tikz}
\usepackage[normalem]{ulem}
\tikzset{
    %Define standard arrow tip
%    >=stealth',
    %Define style for boxes
    oval/.style={
           rectangle,
           rounded corners,
           draw=black, very thick,
           text width=6.5em,
           minimum height=2em,
           text centered},
    % Define arrow style
    arr/.style={
           ->,
           thick,
           shorten <=2pt,
           shorten >=2pt,}
}
\usepackage[noend]{algorithmic}
\usepackage[noend]{algorithm}
\newcommand{\bfor}{{\bf for\ }}
\newcommand{\bthen}{{\bf then\ }}
\newcommand{\bwhile}{{\bf while\ }}
\newcommand{\btrue}{{\bf true\ }}
\newcommand{\bfalse}{{\bf false\ }}
\newcommand{\bto}{{\bf to\ }}
\newcommand{\bdo}{{\bf do\ }}
\newcommand{\bif}{{\bf if\ }}
\newcommand{\belse}{{\bf else\ }}
\newcommand{\band}{{\bf and\ }}
\newcommand{\breturn}{{\bf return\ }}
\newcommand{\mod}{{\rm mod}}
\renewcommand{\algorithmiccomment}[1]{$\rhd$ #1}
\newenvironment{checklist}{\par\noindent\hspace{-.25in}{\bf Checklist:}\renewcommand{\labelitemi}{$\Box$}%
\begin{itemize}}{\end{itemize}}
\pagestyle{threepartheadings}
\usepackage{url}
\usepackage{wrapfig}
% removing the standard hyperref to avoid the horrible boxes
%\usepackage{hyperref}
\usepackage[hidelinks]{hyperref}
% added in the dtklogos for the bibtex formatting
\usepackage{dtklogos}
%=========================
% One-inch margins everywhere
%=========================
\setlength{\topmargin}{0in}
\setlength{\textheight}{8.5in}
\setlength{\oddsidemargin}{0in}
\setlength{\evensidemargin}{0in}
\setlength{\textwidth}{6.5in}
%===============================
%===============================
% Macro for document title:
%===============================
\newcommand{\MYTITLE}[1]%
   {\begin{center}
     \begin{center}
     \bf
     CMPSC 591\\Principles of Mobile Applications\\
     Fall 2013
     \medskip
     \end{center}
     \bf
     #1
     \end{center}
}
%================================
% Macro for headings:
%================================
\newcommand{\MYHEADERS}[2]%
   {\lhead{#1}
    \rhead{#2}
    %\immediate\write16{}
    %\immediate\write16{DATE OF HANDOUT?}
    %\read16 to \dateofhandout
    \def \dateofhandout {October 24 and 25, 2013}
    \lfoot{\sc Handed out on \dateofhandout}
    %\immediate\write16{}
    %\immediate\write16{HANDOUT NUMBER?}
    %\read16 to\handoutnum
    \def \handoutnum {7}
    \rfoot{Handout \handoutnum}
   }

%================================
% Macro for bold italic:
%================================
\newcommand{\bit}[1]{{\textit{\textbf{#1}}}}

%=========================
% Non-zero paragraph skips.
%=========================
\setlength{\parskip}{1ex}

%=========================
% Create various environments:
%=========================
\newcommand{\PURPOSE}{\par\noindent\hspace{-.25in}{\bf Purpose:\ }}
\newcommand{\SUMMARY}{\par\noindent\hspace{-.25in}{\bf Summary:\ }}
\newcommand{\DETAILS}{\par\noindent\hspace{-.25in}{\bf Details:\ }}
\newcommand{\HANDIN}{\par\noindent\hspace{-.25in}{\bf Hand in:\ }}
\newcommand{\SUBHEAD}[1]{\bigskip\par\noindent\hspace{-.1in}{\sc #1}\\}
%\newenvironment{CHECKLIST}{\begin{itemize}}{\end{itemize}}


\usepackage[compact]{titlesec}

\begin{document}
\MYTITLE{Homework Assignment Six: Components and Blocks in AppInventor}
\MYHEADERS{Homework Assignment Six}{Due: October 31 and November 1, 2013}

\vspace*{-.1in}
\section*{Introduction}

In this homework assignment we will use the Component Designer and the Blocks Editor to create a simple mobile app that
can play sounds and perform actions when you either touch or shake the tablet.  As in the previous homework assignment,
you should take time to learn more about AppInventor and get ready to program your app by visiting the Web site
\url{http://appinventor.mit.edu/}. Students who want to learn more about the AppInventor system or download a PDF of
today's printed chapter should also visit \url{http://www.appinventor.org/}.

\section*{Using the Component Designer and the Blocks Editor}

Following the tutorial in the handout provided by the instructor, use the Component Designer and the Blocks Editor
to create an app that displays an image and reacts when the user manipulates the tablet.  You should use the Component
Designer when you want to add both visible (e.g., buttons and labels) and non-visible (e.g., sounds and sensors)
components to your mobile app.  Using the Component Designer involves dragging components from the list of those
available to the canvas of your mobile app.  You can use the Blocks Editor when you want to program the logic that
describes how the user interface components will behave when subject to interaction. Using the blocks editor involves
the assembly of ``puzzle pieces'' that describe your mobile app's behavior.

After you have reviewed the printed version of Chapter 1 in the AppInventor text book, you should implement the
described application, adding your own extensions in several areas.  For instance, instead of adding the picture of a
cat, you should pick a different picture.  Then, you should select sounds that will work well with your chosen picture.
Finally, you should implement one or two additional features by adding components and blocks that are not specifically
mentioned in Chapter 1 of the AppInventor book. Once you have completed your app, you should package it for download to
the Nexus 7 tablet.  Before packaging the app, please make sure that you have designed a high-resolution icon that will
be associated with your app when it is run on the tablet.

Does your app work correctly when your tablet is not connected to the development workstation by a USB cable? Now that
you have finished implementing your first app in AppInventor, please take some time to reflect on your experiences.
What are the strengths and weaknesses associated with using AppInventor? What are three apps that you could implement in
AppInventor?

% \section*{Summary of the Required Deliverables}

\vspace*{.05in}
\noindent
To complete the assignment, you should turn in one copy of the following signed printouts: 
\vspace*{-.06in}

\begin{enumerate}
	\itemsep0em
	\item A screenshot of the Component Designer with all of the components in the completed app
	\item A screenshot of the Blocks Editor with all of the logic blocks in the completed app 
	\item Screenshots showing both the app's icon and the main screen when it runs stand-alone
	\item A discussion of the strengths and weaknesses associated with app development in AppInventor
	\item A description of three Android apps that you could feasibly implement in AppInventor
\end{enumerate}

% \begin{enumerate}
%     \itemsep0em
% 
% 	\item Description of at least three automation tasks that your team could implement
% 		
% 	\item Complete documentation that describes the following aspects of your chosen automation tasks
% 
% \vspace*{-.05in}
% \begin{enumerate}
% 	\item Motivation for implementing the chosen tasks
% 	\item Inputs, outputs, and behavior of the tasks
% 	\item Implementation and testing choices for the tasks
% \end{enumerate}
% \vspace*{-.05in}
% 
% 	\item Report on the results from the user study that you performed with your automation tasks
% 
% 	\item Presentation slides for the in-class talk and demonstration that you will give in two weeks
% 
% \end{enumerate}
% 
\end{document}
