% CS 580 style
% Typical usage (all UPPERCASE items are optional):
%       \input 580pre
%       \begin{document}
%       \MYTITLE{Title of document, e.g., Lab 1\\Due ...}
%       \MYHEADERS{short title}{other running head, e.g., due date}
%       \PURPOSE{Description of purpose}
%       \SUMMARY{Very short overview of assignment}
%       \DETAILS{Detailed description}
%         \SUBHEAD{if needed} ...
%         \SUBHEAD{if needed} ...
%          ...
%       \HANDIN{What to hand in and how}
%       \begin{checklist}
%       \item ...
%       \end{checklist}
% There is no need to include a "\documentstyle."
% However, there should be an "\end{document}."
%
%===========================================================
\documentclass[11pt,twoside,titlepage]{article}
%%NEED TO ADD epsf!!
\usepackage{threeparttop}
\usepackage{graphicx}
\usepackage{latexsym}
\usepackage{color}
\usepackage{listings}
\usepackage{fancyvrb}
%\usepackage{pgf,pgfarrows,pgfnodes,pgfautomata,pgfheaps,pgfshade}
\usepackage{tikz}
\usepackage[normalem]{ulem}
\tikzset{
    %Define standard arrow tip
%    >=stealth',
    %Define style for boxes
    oval/.style={
           rectangle,
           rounded corners,
           draw=black, very thick,
           text width=6.5em,
           minimum height=2em,
           text centered},
    % Define arrow style
    arr/.style={
           ->,
           thick,
           shorten <=2pt,
           shorten >=2pt,}
}
\usepackage[noend]{algorithmic}
\usepackage[noend]{algorithm}
\newcommand{\bfor}{{\bf for\ }}
\newcommand{\bthen}{{\bf then\ }}
\newcommand{\bwhile}{{\bf while\ }}
\newcommand{\btrue}{{\bf true\ }}
\newcommand{\bfalse}{{\bf false\ }}
\newcommand{\bto}{{\bf to\ }}
\newcommand{\bdo}{{\bf do\ }}
\newcommand{\bif}{{\bf if\ }}
\newcommand{\belse}{{\bf else\ }}
\newcommand{\band}{{\bf and\ }}
\newcommand{\breturn}{{\bf return\ }}
\newcommand{\mod}{{\rm mod}}
\renewcommand{\algorithmiccomment}[1]{$\rhd$ #1}
\newenvironment{checklist}{\par\noindent\hspace{-.25in}{\bf Checklist:}\renewcommand{\labelitemi}{$\Box$}%
\begin{itemize}}{\end{itemize}}
\pagestyle{threepartheadings}
\usepackage{url}
\usepackage{wrapfig}
% removing the standard hyperref to avoid the horrible boxes
%\usepackage{hyperref}
\usepackage[hidelinks]{hyperref}
% added in the dtklogos for the bibtex formatting
\usepackage{dtklogos}
%=========================
% One-inch margins everywhere
%=========================
\setlength{\topmargin}{0in}
\setlength{\textheight}{8.5in}
\setlength{\oddsidemargin}{0in}
\setlength{\evensidemargin}{0in}
\setlength{\textwidth}{6.5in}
%===============================
%===============================
% Macro for document title:
%===============================
\newcommand{\MYTITLE}[1]%
   {\begin{center}
     \begin{center}
     \bf
     CMPSC 290\\Principles of Software Development\\
     Fall 2013
     \medskip
     \end{center}
     \bf
     #1
     \end{center}
}
%================================
% Macro for headings:
%================================
\newcommand{\MYHEADERS}[2]%
   {\lhead{#1}
    \rhead{#2}
    %\immediate\write16{}
    %\immediate\write16{DATE OF HANDOUT?}
    %\read16 to \dateofhandout
    \def \dateofhandout {August 27, 2013}
    \lfoot{\sc Handed out on \dateofhandout}
    %\immediate\write16{}
    %\immediate\write16{HANDOUT NUMBER?}
    %\read16 to\handoutnum
    \def \handoutnum {1}
    \rfoot{Handout \handoutnum}
   }

%================================
% Macro for bold italic:
%================================
\newcommand{\bit}[1]{{\textit{\textbf{#1}}}}

%=========================
% Non-zero paragraph skips.
%=========================
\setlength{\parskip}{1ex}

%=========================
% Create various environments:
%=========================
\newcommand{\PURPOSE}{\par\noindent\hspace{-.25in}{\bf Purpose:\ }}
\newcommand{\SUMMARY}{\par\noindent\hspace{-.25in}{\bf Summary:\ }}
\newcommand{\DETAILS}{\par\noindent\hspace{-.25in}{\bf Details:\ }}
\newcommand{\HANDIN}{\par\noindent\hspace{-.25in}{\bf Hand in:\ }}
\newcommand{\SUBHEAD}[1]{\bigskip\par\noindent\hspace{-.1in}{\sc #1}\\}
%\newenvironment{CHECKLIST}{\begin{itemize}}{\end{itemize}}


\usepackage[compact]{titlesec}

\begin{document}
\MYTITLE{Syllabus}
\MYHEADERS{Syllabus}{}

\subsection*{Course Instructor}
Dr.\ Gregory M.\ Kapfhammer\\
\noindent Office Location: Alden Hall 108 \\
\noindent Office Phone: +1 814-332-2880 \\
\noindent Email: \url{gkapfham@allegheny.edu} \\
\noindent Twitter: \url{@GregKapfhammer} \\
\noindent Web Site: \url{http://www.cs.allegheny.edu/sites/gkapfham/} 

\subsection*{Instructor's Office Hours}

\begin{itemize}
	\itemsep 0em
	\item Monday: 1:00 pm -- 2:30 pm (30 minute time slots)
	\item Tuesday: 2:30 pm -- 4:00 pm (15 minute time slots)
	\item Wednesday: 4:30 pm -- 5:00 pm (15 minute time slots)
	\item Thursday: 9:00 am -- 10:00 am (15 minute time slots) {\em and} \\ \hspace*{.69in} 2:30 pm -- 4:00 pm (15 minute time slots)
	\item Friday: 1:00 pm -- 2:30 pm (10 minute time slots) {\em and} \\ \hspace*{.49in} 4:30 pm -- 5:00 pm (5 minute time slots)
\end{itemize}

\noindent
To schedule a meeting with me during my office hours, please visit my Web site and click the ``Schedule'' link in the
top right-hand corner. Now, you can browse my office hours or schedule an appointment by clicking the correct link and
then reserving an open time slot. 

\subsection*{Course Meeting Schedule}

Discussion, Presentations, and Group Work: \\ Thursday, 1:30 pm -- 2:20 pm and/or Friday, 3:30 pm - 4:20 pm 

\subsection*{Course Description}

\begin{quote}

	A study of the concepts, principles, and skills needed to successfully describe, design, implement, test, deploy,
	use, and document mobile applications.  Investigated in the context of Android-based mobile applications running on
	the Google Nexus 7, topics include conditional logic, iteration, modularity, parameter passing, user interaction,
	graphical user interfaces, and network communication. Students practice the principles of mobile application
	development by participating as members of groups tasked with the creation of mobile applications.  One required
	class session per week. No background in computer science is required. Prerequisites: Permission of the instructor.
	\\ {\em Two credit group study}.
	
\end{quote}

\subsection*{Course Objectives}

The process of developing mobile software involves the application of a number of interesting concepts, tools,
techniques, and methodologies.  In this class we will explore the steps taken to develop mobile applications and examine
the tools, concepts, and challenges associated with each step.  We will delve into the details of describing, designing,
implementing, testing, deploying, using, and documenting mobile applications through a discussion of book chapters and
articles from the literature on mobile application development.  Moreover, students will enhance their ability to
clearly and concisely write about software in general and mobile applications in specific.  Finally, students will gain
practical software development experience through group projects that yield a wide variety of different mobile
applications. Students will complete a final project with the ultimate goal of releasing a mobile application on the
Google Play store

\subsection*{Performance Objectives}

At the completion of this class, a student should be aware of the fundamental challenges associated with the development 
of mobile applications.  Furthermore, students should be familiar with a wide array of concepts, methodologies, techniques, and
tools that they can apply to the problem of developing simple mobile applications.  However, a successful student will
emerge with more than an understanding of the tools (e.g., text editors, compilers, debuggers, and integrated development
environments) that a mobile application developer uses.  A student also should have a basic understanding of the steps
that must be taken to develop a mobile application for the Android operating system.  During the process of mobile
application development, the student should also be able to work in a group and interact with prospective users of their
mobile application.

\subsection*{Required Textbooks}

This course does not have a required textbook.  Instead, the instructor will prepare an annotated reading list and release it on
the course Web site. Students who want to learn more about the general principles of software development may consult the following book.

% Shari Lawrence Pfleeger and Joanne M. Atlee
%   Software Engineering: Theory and Practice (Fourth Edition)
%   ISBN-10: 0136061699
%   ISBN-13: 978-0136061694
%   Prentice Hall
%   Status: Required
%   25 copies

\noindent{\em Software Engineering: Theory and Practice}. Shari Lawrence Pfleeger and Joanne M. Atlee.
Fourth Edition, ISBN-10: 0136061699, ISBN-13: 978-0136061694, 792 pages, 2009. 

\noindent
Students who want to improve their technical writing skills may consult the following books.

\noindent{\em BUGS in Writing: A Guide to Debugging Your Prose}. Lyn Dupr\'e. Second Edition,  ISBN-10: 020137921X,
ISBN-13: 978-0201379211, 704 pages, 1998.

\noindent{\em Writing for Computer Science}.  Justin Zobel. Second Edition,  ISBN-10: 1852338024, ISBN-13:
978-1852338022, 270 pages, 2004.

\subsection*{Class Policies}

\subsubsection*{Grading}

The grade that a student receives in this class will be based on the
following categories. All percentages are approximate and it is possible
for the assigned percentages to change during the academic semester if a
need to do so presents itself. 

\begin{center}
\begin{tabular}{llll}
Class Participation & 10\% & Homework Assignments & 60\% \\
Instructor Meetings & 10\% & Final Project & 20\%
\end{tabular}
\end{center}

\vspace*{-.1in}
\noindent
These grading categories have the following definitions:
\vspace*{-.1in}


\begin{itemize}

	\item {\em Class Participation and Instructor Meetings}: All students are required to actively participate during
		all of the class sessions. Your participation will take forms such as answering questions about the required
		reading assignments, asking constructive questions of your group members, giving presentations, and leading a
		discussion session. Furthermore, all students are required to meet with the course instructor during office
		hours for a total of sixty minutes during the Fall 2013 semester.  These meetings must be scheduled through the
		course instructor's reservation system and documented on a meeting record that you submit on the day of the final
		examination. A student will receive an interim and final grade for these categories.

	\item {\em Homework Assignments}: These assignments invite students to explore the concepts, tools, and techniques
		that are associated with the development of mobile applications.  All of the homework assignments require the
		use of the provided tools to describe, design, implement, test, deploy, use, and document mobile applications.
		To ensure that students are ready to develop software in both other classes at Allegheny College and after
		graduation, the instructor will assign individuals to teams for each of the homework assignments.  Unless
		specified otherwise, each homework assignment will be due at the beginning of next week's class session.  Many of
		the homework assignments in this course will invite students to give both a presentation and a demonstration of
		the software that they described, designed, implemented, tested, and documented. Often, students will also be
		asked to survey potential users about the strengths and weaknesses of their mobile applications.
	
	\item {\em Final Project}: This project will afford you the opportunity to complete a useful and interesting mobile
		application with the intention of releasing it on the Google Play store.	The final project in this class will
		require you to apply all of the knowledge and skills that you have accumulated throughout the semester
		to finish a working Android application and, whenever possible, make it publicly available.  The project will
		require you to draw upon both your problem solving skills and your knowledge of programming languages and tools
		that support the development of mobile applications. The final project will be completed in groups chosen in
		consultation with the course instructor.
		
\end{itemize}

\subsubsection*{Assignment Submission}

All assignments will have a stated due date. The printed version of the assignment is to be turned in at the beginning
of the class on that due date; the printed materials must be dated and signed with the Honor Code pledge of all the
students in the group.  The electronic version of the assignment must be made available to the course instructor when
the printed version is submitted. Late assignments will be accepted for up to one week past the assigned due date with a
10\% penalty. All late assignments must be submitted at the beginning of the session that is scheduled one week after
the due date. Unless special arrangements are made with the course instructor, no assignments will be accepted after the
late deadline. In addition to submitting the required deliverables for an assignment developed in a group, students must
turn in a one-page document that describes each group member's contribution to the completion of the submitted deliverables.  

\subsubsection*{Attendance}

Students may choose to attend one or both of the Thursday and Friday class sessions.  However, students attending the
Thursday session should anticipate covering similar material during Friday's class.  It is mandatory for all students to
attend at least one class session every week. If you will not be able to attend a session, then please see the course
instructor at least one week in advance to describe your situation.  Students who miss more than five unexcused classes
sessions or group project meetings will have their final grade in the course reduced by one letter grade.  Students who
miss more than ten of the aforementioned events will automatically fail the course.

\subsubsection*{Use of Laboratory Facilities}

Throughout the semester, we will experiment with many different tools that mobile application developers use.  The
course instructor and the department's systems administrator have invested a considerable amount of time to ensure that
our laboratories support the completion of both the laboratory assignments and the final project.  To this end, students
are required to complete all assignments and the final project while using the department's laboratory facilities. The
course instructor and the systems administrator will only be able to devote a limited amount of time to the
configuration of a student's personal computer or mobile device. Finally, students must take great care when using the
mobile devices --- you should not store personal data or files on them and you should use them carefully to avoid
causing any damage.

\subsubsection*{Class Preparation}

In order to minimize confusion and maximize learning, students must invest time to prepare for the class sessions.
During the class periods, the course instructor will often pose demanding questions that could require group discussion,
the creation of a mobile application, a vote on a thought-provoking issue, or a group presentation.  Only students who
have prepared for class by reading the assigned material and reviewing the both current and recent assignments will be
able to effectively participate in these discussions.  More importantly, only prepared students will be able to acquire
the knowledge and skills that they need to be successful in both this course and the field of mobile application
development.  In order to help students remain organized and effectively prepare for classes, the course instructor will
maintain a class schedule with reading assignments and presentation slides.   During the class sessions students will
also be required to download, use, and modify programs that are made available through the course Web site.

\subsubsection*{Email}

Using your Allegheny College email address, I will sometimes send out class announcements about matters such as
assignment clarifications or changes in the schedule. It is your responsibility to check your email at least once a day
and to ensure that you can reliably send and receive emails. This class policy is based on the following statement in
{\em The Compass}, the college's student handbook.

\vspace*{-.1in}
\begin{quote}
``The use of email is a primary method of communication on campus. \ldots
All students are provided with a campus email account and address while
enrolled at Allegheny and are expected to check the account on a regular
basis.'' 
\end{quote}
\vspace*{-.15in}

\subsubsection*{Disability Services}

The Americans with Disabilities Act (ADA) is a federal anti-discrimination statute that provides comprehensive civil
rights protection for persons with disabilities.  Among other things, this legislation requires all students with
disabilities be guaranteed a learning environment that provides for reasonable accommodation of their disabilities.
Students with disabilities who believe they may need accommodations in this class are encouraged to contact Disability
Services at 332-2898.  Disability Services is part of the Learning Commons and is located in Pelletier Library.
Please do this as soon as possible to ensure that approved accommodations are implemented in a timely fashion.

\subsubsection*{Honor Code}

The Academic Honor Program that governs the entire academic program at Allegheny College is described in the Allegheny
Course Catalogue.  The Honor Program applies to all work that is submitted for academic credit or to meet non-credit
requirements for graduation at Allegheny College.  This includes all work assigned for this class (e.g., homework
assignments and the final project).  All students who have enrolled in the College will work under the Honor Program.
Each student who has matriculated at the College has acknowledged the following pledge:

\vspace*{-.1in}
\begin{quote}
I hereby recognize and pledge to fulfill my responsibilities, as defined in the Honor Code, and to maintain the
integrity of both myself and the College community as a whole.  
\end{quote}
\vspace*{-.15in}

\subsection*{Welcome to an Adventure in Mobile Application Development}

In reference to software, Frederick Brooks, Jr.\ wrote in Chapter One of {\em The Mythical Man Month}, ``The magic of
myth and legend has come true in our time.'' Computer software --- especially software delivered as a mobile application
--- is a pervasive aspect of our society that changes how we think and act.  High quality mobile applications also have
the potential to positively influence the lives of people. Moreover, the description, design, implementation, testing,
deployment, use, and documentation of mobile software are exciting and rewarding activities!  At the start of this class, I
invite you to pursue with enthusiasm and vigor this adventure in mobile application development.

\end{document}
